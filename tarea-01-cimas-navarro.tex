\documentclass[12pt]{article}

\usepackage[a4paper, top=2cm]{geometry}
\usepackage{pifont}
\usepackage{xfp}
\usepackage[dvipsnames]{xcolor}
\usepackage{titlesec}
\usepackage[most]{tcolorbox}
\usepackage{csvsimple}
\usepackage{luacode}

\pagenumbering{gobble}

\newcommand{\sectionfont}{\Large\bfseries}

\titleformat{\section}
    {\titlerule
     \vspace{0.5ex}%
     \sectionfont}
    {\thesection}{1em}
    {\sectionfont}[\titlerule]

\newtcolorbox{inspirational}{enhanced, interior hidden, frame code={
  \draw[rounded corners, thick] (interior.north west)++ (0,-0.5) -- ++(0,0.5) -- ++(0.5,0) ;
  \draw[rounded corners, thick] (interior.south east)++ (0,0.5) -- ++(0,-0.5) -- ++(-0.5,0) ;
}}

\begin{filecontents*}[overwrite]{students-list.csv}
Alberto Romero Tomás,romero-tomas
Andreu Buenafuente Moreno,buenafuente-moreno
Raúl Cimas Navarro,cimas-navarro
\end{filecontents*}

\begin{luacode*}
    n = 0
    t = 0.0

    function points(p)

        local omark = [[{\color{OliveGreen}\ding{52}}]]
        local xmark = [[{\color{Bittersweet}\ding{56}}]]
        local pts = "("..p.." pts)"

        -- I'd prefer these to be outside the function
        n = n + 1
        t = t + p

        if p >= 1.0 then
            tex.print("[ "..omark.." ] "..pts.." " )
        else
            tex.print("[ "..xmark.." ] "..pts.." " )
        end
    end

    function getgrade()
        local g = math.floor(t / n * 100)
        tex.print(g)
    end

    function getname(s)
        tex.print(string.match(tostring(s), "[^tarexmn%p%d][%a%p]*"))
    end

    function getnumber(s)
        tex.print(tonumber(string.match(s, "%d%d")))
    end

    function gettype(s)
        if string.match(s, "tarea") == "tarea" then
            tex.print("Tarea")
        else
            tex.print("Examen")
        end
    end

\end{luacode*}


\newcommand{\point}[1]{
    \item\textbf{\directlua{points(#1)}}
}

\newcommand{\getname}[1]{
    \csvreader[no head, filter equal={\csvcolii}{\directlua{getname(#1)}}]{./students-list.csv}{}{\csvcoli}
}

\newcommand{\getnumber}[1]{
    \directlua{getnumber(#1)}
}

\begin{document}

\title{
    \Large{\textsc{Heurística II}} \\
%    \Large{\grade{4.75}{5}}
}
\author{}
\date{}

\maketitle

\section*{Comentarios}

\begin{enumerate}
        \point{1.0} Un poco rara la gráfica... Y para encontrar la esfuera
        pudiste aprovechar la simetría axial y ahorrarte la $z$.
        \point{1.0} ¿Por qué no usar el teorema de la divergencia?
        \point{0.75} Está mal planteada la integral.
        $\hat{\rho} = \cos\theta\hat{i} + \sin\theta\hat{j}$ por lo que
        hay una dependencia de $\theta$ que no consideraste.
        \point{1.0} Creo que la notación se pudo simplificar un poco.
        \point{1.0} % Revisado
\end{enumerate}

\section*{Testing}

Tarea\getnumber{"tarea-01-cimas-navarro"}de
\getname{"tarea-01-cimas-navarro"}

This file is named: \jobname

\directlua{
    getgrade()
    gettype("tarea-01-cimas-navarro")
}

\vspace*{\fill}

%\begin{inspirational}
%\input{../phrase.tex}
%\end{inspirational}

\end{document}
