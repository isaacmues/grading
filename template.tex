\documentclass[12pt]{article}

\usepackage[a4paper, top=2cm]{geometry}
\usepackage{ifthen}
\usepackage{xcolor}
\usepackage{pifont}


\newcommand{\studentName}[1]{
    \ifthenelse{\equal{estudiante_prueba}{#1}}{Estudiante Prueba}{}
}

\newcommand{\correct}[1]{
    \item\textbf{[ \ding{52} ] (#1 pts)}
}

\newcommand{\incorrect}[1]{
    \item\textbf{[ \ding{56} ] (#1 pts)}
}

\newcommand{\point}[1]{
    \ifthenelse{\equal{1.0}{#1}}{\correct{#1}}{\incorrect{#1}}
}


\begin{document}

\title{
    Tarea de \studentName{estudiante_prueba} \\
    \Large{\textsc{className}}
}
\author{}
\date{}

\maketitle

\section*{Comentarios}

\begin{center}
    \rule{0.8\textwidth}{1pt}
\end{center}

\begin{enumerate}
        \point{0.25}
        \point{1.0}
\end{enumerate}

\end{document}
